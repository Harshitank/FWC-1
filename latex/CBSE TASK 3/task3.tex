\documentclass[12pt,fleqn]{article}

\usepackage{amsmath}
\usepackage{xcolor}
\usepackage{graphicx}
\usepackage[none]{hyphenat}
\usepackage{enumitem}

\setlength{\parindent}{0pt}
\setlength{\mathindent}{0pt}

\definecolor{mycyan}{RGB}{0,153,204}

\setlist[enumerate]{label=\textcolor{mycyan}{\arabic*.}, leftmargin=1.2cm}


\begin{document}


%---------------- HEADER ----------------
\begin{center}
\includegraphics[width=4cm]{iiitb_logo.png}\\[4pt]
\textbf{Harshita N Kumar}\\
ID: COMETFWC052\\
CBSE Class XII
\end{center}

\vspace{8pt}


\begin{center}
\fcolorbox{mycyan}{mycyan!20}{
\parbox{0.9\textwidth}{\centering \textbf{EXERCISE 4.6}}
}
\end{center}


Examine the consistency of the system of equations in Exercises 1 to 6.
\begin{enumerate}
\item
$x + 2y = 2$\\
$2x + 3y = 3$
\item
$2x - y = 5$\\
$x + y = 4$
\item
$x + 3y = 5$\\
$2x + 6y = 8$
\item
$x + y + z = 1$\\
$2x + 3y + 2z = 2$\\
$ax + ay + 2az = 4$
\item
$3x - y - 2z = 2$\\
$2y - z = -1$\\
$3x - 5y = 3$
\item
$5x - y + 4z = 5$\\
$2x + 3y + 5z = 2$\\
$5x - 2y + 6z = -1$
\end{enumerate}
Solve system of linear equations using matrix method in Exercises 7 to 14.
\begin{enumerate}
\setcounter{enumi}{6}
\item
$5x + 2y = 4$\\
$7x + 3y = 5$
\item
$2x - y = -2$\\
$3x + 4y = 3$
\item
$4x - 3y = 3$\\
$3x - 5y = 7$
\item
$5x + 2y = 3$\\
$3x + 2y = 5$
\item
$2x + y + z = 1$\\
$x - 2y - z = \dfrac{3}{2}$\\
$3y - 5z = 9$
\item
$x - y + z = 4$\\
$2x + y - 3z = 0$\\
$x + y + z = 2$
\item
$2x + 3y + 3z = 5$\\
$x - 2y + z = -4$\\
$3x - y - 2z = 3$
\item
$x - y + 2z = 7$\\
$3x + 4y - 5z = -5$\\
$2x - y + 3z = 12$
\end{enumerate}
\textcolor{cyan}{15.} If
$
A=
\begin{pmatrix}
2 & -3 & 5\\
3 & 2 & -4\\
1 & 1 & -2
\end{pmatrix}
$,
find $A^{-1}$. Using $A^{-1}$ solve
\begin{align*}
2x - 3y + 5z &= 11\\
3x + 2y - 4z &= -5\\
x + y - 2z &= -3
\end{align*}
\par\vspace{6pt}   % <<< THIS IS THE KEY LINE
\noindent\textcolor{cyan}{16.}
The cost of $4$ kg onion, $3$ kg wheat and $2$ kg rice is Rs.\ $60$.\\
The cost of $2$ kg onion, $4$ kg wheat and $6$ kg rice is Rs.\ $90$.\\
The cost of $6$ kg onion, $2$ kg wheat and $3$ kg rice is Rs.\ $70$.\\
Find the cost of each item per kg by matrix method.




\newpage
\begin{center}
\fcolorbox{cyan}{cyan!20}{
\parbox{0.9\textwidth}{

\begin{center}
\textbf{\large Summary}
\end{center}

\vspace{0.3cm}

\begin{itemize}

\item Determinant of a matrix $A=[a_{11}]_{1\times1}$ is given by  
$|a_{11}| = a_{11}$.

\item Determinant of a matrix  
$A=
\begin{pmatrix}
a_{11} & a_{12}\\
a_{21} & a_{22}
\end{pmatrix}$  
is given by  
$|A|=
\begin{vmatrix}
a_{11} & a_{12}\\
a_{21} & a_{22}
\end{vmatrix}
= a_{11}a_{22}-a_{12}a_{21}$.

\item Determinant of a matrix  
$A=
\begin{pmatrix}
a_1 & b_1 & c_1\\
a_2 & b_2 & c_2\\
a_3 & b_3 & c_3
\end{pmatrix}$  
is given by (expanding along $R_1$)  

$|A|
= a_1
\begin{vmatrix}
b_2 & c_2\\
b_3 & c_3
\end{vmatrix}
- b_1
\begin{vmatrix}
a_2 & c_2\\
a_3 & c_3
\end{vmatrix}
+ c_1
\begin{vmatrix}
a_2 & b_2\\
a_3 & b_3
\end{vmatrix}$.

\item For any square matrix $A$, the determinant $|A|$ satisfies the following properties.

\item $|A'| = |A|$, where $A'$ is the transpose of $A$.

\item If we interchange any two rows (or columns), then the sign of determinant changes.

\item If any two rows or any two columns are identical or proportional, then the value of determinant is zero.

\item If we multiply each element of a row or a column of a determinant by a constant $k$, then the value of determinant is multiplied by $k$.

\item Multiplying a determinant by $k$ means multiplying elements of only one row (or one column) by $k$.

\item If $A=[a_{ij}]_{3\times3}$, then  
$|kA| = k^3|A|$.

\item If elements of a row or a column of a determinant can be expressed as sum of two or more elements, then the determinant can be expressed as sum of two or more determinants.

\item If to each element of a row or a column of a determinant the equimultiples of corresponding elements of other rows or columns are added, then the value of determinant remains same.

\end{itemize}

}
}


\end{center}

\newpage
\begin{center}
\fcolorbox{cyan}{cyan!20}{
\parbox{0.9\textwidth}{

\vspace{0.2cm}

\begin{itemize}

\item Area of a triangle with vertices $(x_1,y_1)$, $(x_2,y_2)$ and $(x_3,y_3)$ is given by  
$\Delta=\dfrac{1}{2}
\begin{vmatrix}
x_1 & y_1 & 1\\
x_2 & y_2 & 1\\
x_3 & y_3 & 1
\end{vmatrix}$.

\item Minor of an element $a_{ij}$ of the determinant of matrix $A$ is the determinant obtained by deleting $i^{th}$ row and $j^{th}$ column and is denoted by $M_{ij}$.

\item Cofactor of $a_{ij}$ is given by  
$A_{ij}=(-1)^{i+j}M_{ij}$.

\item Value of determinant of a matrix $A$ is obtained by sum of product of elements of a row (or a column) with corresponding cofactors.  
For example,  
$|A| = a_{11}A_{11} + a_{12}A_{12} + a_{13}A_{13}$.

\item If elements of one row (or column) are multiplied with cofactors of elements of any other row (or column), then their sum is zero.  
For example,  
$a_{11}A_{21} + a_{12}A_{22} + a_{13}A_{23} = 0$.

\item If  
$A=
\begin{pmatrix}
a_{11} & a_{12} & a_{13}\\
a_{21} & a_{22} & a_{23}\\
a_{31} & a_{32} & a_{33}
\end{pmatrix}$,  
then  
$adj\,A=
\begin{pmatrix}
A_{11} & A_{21} & A_{31}\\
A_{12} & A_{22} & A_{32}\\
A_{13} & A_{23} & A_{33}
\end{pmatrix}$,  
where $A_{ij}$ is the cofactor of $a_{ij}$.

\item $A(adj\,A) = (adj\,A)A = |A|I$, where $A$ is a square matrix of order $n$.

\item A square matrix $A$ is said to be singular or non-singular according as $|A| = 0$ or $|A| \neq 0$.

\item If $AB = BA = I$, where $B$ is a square matrix, then $B$ is called inverse of $A$.  
Also $A^{-1} = B$ or $B^{-1} = A$ and hence $(A^{-1})^{-1} = A$.

\item A square matrix $A$ has inverse if and only if $A$ is non-singular.

\item $A^{-1} = \dfrac{1}{|A|}(adj\,A)$.

\item If  
$a_1x + b_1y + c_1z = d_1$,  
$a_2x + b_2y + c_2z = d_2$,  
$a_3x + b_3y + c_3z = d_3$,  
then these equations can be written as $AX = B$, where  

$A=
\begin{pmatrix}
a_1 & b_1 & c_1\\
a_2 & b_2 & c_2\\
a_3 & b_3 & c_3
\end{pmatrix}$,  
$X=
\begin{pmatrix}
x\\
y\\
z
\end{pmatrix}$ and  
$B=
\begin{pmatrix}
d_1\\
d_2\\
d_3
\end{pmatrix}$.

\end{itemize}

\vspace{0.2cm}

}
}
\end{center}

\newpage
\begin{center}
\fcolorbox{cyan}{cyan!20}{
\parbox{0.9\textwidth}{

\vspace{0.2cm}

\begin{itemize}

\item Unique solution of equation $AX = B$ is given by  
$X = A^{-1}B$, where $|A| \neq 0$.

\item A system of equation is consistent or inconsistent according as its solution exists or not.

\item For a square matrix $A$ in matrix equation $AX = B$:

\begin{itemize}
\item[(i)] If $|A| \neq 0$, there exists a unique solution.
\item[(ii)] If $|A| = 0$ and $(adj\,A)B \neq 0$, then there exists no solution.
\item[(iii)] If $|A| = 0$ and $(adj\,A)B = 0$, then the system may or may not be consistent.
\end{itemize}

\end{itemize}

\vspace{0.2cm}

}
}
\end{center}

\end{document}
