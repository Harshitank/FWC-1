\documentclass[12pt]{article}

\usepackage{amsmath}
\usepackage{graphicx}
\usepackage{xcolor}
\usepackage[most]{tcolorbox}
\usepackage[none]{hyphenat}

\setlength{\parindent}{0pt}

% ---- COLOR DEFINITION (FIX) ----
\definecolor{summaryblue}{RGB}{220,235,255}

\begin{document}


\begin{center}
\includegraphics[width=4.5cm]{iiitb_logo.png}\\
{\Large Harshita N Kumar}\\
ID: COMETFWC052\\
CBSE Class XII\\
Task 3
\end{center}

\begin{center}
\fcolorbox{cyan}{cyan!20}{
\parbox{0.9\textwidth}{
\centering
\textbf{EXERCISE 4.6}
}}
\end{center}

Examine the consistency of the system of equations in Exercises 1 to 6.

\begin{enumerate}
\item $x+2y=2$\\
$2x+3y=3$

\item $2x-y=5$\\
$x+y=4$

\item $x+3y=5$\\
$2x+6y=8$

\item $x+y+z=1$\\
$2x+3y+2z=2$\\
$ax+ay+2az=4$

\item $3x-y-2z=2$\\
$2y-z=-1$\\
$3x-5y=3$

\item $5x-y+4z=5$\\
$2x+3y+5z=2$\\
$5x-2y+6z=-1$
\end{enumerate}

Solve system of linear equations using matrix method in Exercises 7 to 14.

\begin{enumerate}
\setcounter{enumi}{6}

\item $5x+2y=4$\\
$7x+3y=5$

\item $2x-y=-2$\\
$3x+4y=3$

\item $4x-3y=3$\\
$3x-5y=7$

\item $5x+2y=3$\\
$3x+2y=5$

\item $2x+y+z=1$\\
$x-2y-z=\frac{3}{2}$\\
$3y-5z=9$

\item $x-y+z=4$\\
$2x+y-3z=0$\\
$x+y+z=2$

\item $2x+3y+3z=5$\\
$x-2y+z=-4$\\
$3x-y-2z=3$

\item $x-y+2z=7$\\
$3x+4y-5z=-5$\\
$2x-y+3z=12$
\end{enumerate}

\bigskip

\textbf{15.} If  
$A=
\begin{pmatrix}
2 & -3 & 5\\
3 & 2 & -4\\
1 & 1 & -2
\end{pmatrix}$,
find $A^{-1}$. Using $A^{-1}$ solve  
$2x-3y+5z=11$\\
$3x+2y-4z=-5$\\
$x+y-2z=-3$

\bigskip

\textbf{16.}  
The cost of $4$ kg onion, $3$ kg wheat and $2$ kg rice is Rs. $60$.  
The cost of $2$ kg onion, $4$ kg wheat and $6$ kg rice is Rs. $90$.  
The cost of $6$ kg onion, $2$ kg wheat and $3$ kg rice is Rs. $70$.  
Find the cost of each item per kg by matrix method.

\newpage
\begin{tcolorbox}[
colback=cyan!20,
colframe=cyan!60!black,
sharp corners,
boxrule=0.8pt,
left=18pt,right=18pt,top=12pt,bottom=12pt
]
\small
\begin{center}
\textbf{Summary}
\end{center}
\begin{itemize}
\item Determinant of a matrix $A=[a_{11}]_{1\times1}$ is given by $|a_{11}|=a_{11}$.
\item Determinant of a matrix
$A=\begin{pmatrix} a_{11}&a_{12}\\ a_{21}&a_{22}\end{pmatrix}$
is given by $|A|=a_{11}a_{22}-a_{12}a_{21}$.
\item Determinant of a matrix
$A=\begin{pmatrix}
a_1 & b_1 & c_1\\
a_2 & b_2 & c_2\\
a_3 & b_3 & c_3
\end{pmatrix}$
(expanding along $R_1$) is
$|A|=a_1\begin{vmatrix} b_2&c_2\\ b_3&c_3\end{vmatrix}
- b_1\begin{vmatrix} a_2&c_2\\ a_3&c_3\end{vmatrix}
+ c_1\begin{vmatrix} a_2&b_2\\ a_3&b_3\end{vmatrix}$.
\item For any square matrix $A$, the determinant $|A|$ satisfies the following properties.
\item $|A'|=|A|$, where $A'$ is the transpose of $A$.
\item Interchanging any two rows (or columns) changes the sign of determinant.
\item If any two rows or columns are identical or proportional, then the determinant is zero.
\item If each element of a row or column is multiplied by $k$, then the determinant is multiplied by $k$.
\item Multiplying a determinant by $k$ means multiplying elements of only one row (or one column) by $k$.
\item If $A=[a_{ij}]_{3\times3}$, then $|kA|=k^3|A|$.
\item If elements of a row or column are sums, the determinant is the sum of determinants.
\item Adding equimultiples of other rows/columns to a row/column does not change the determinant.
\end{itemize}
\end{tcolorbox}
\newpage
\begin{tcolorbox}[
colback=cyan!20,
colframe=cyan!60!black,
sharp corners,
boxrule=0.8pt,
left=18pt,right=18pt,top=12pt,bottom=12pt
]
\small
\begin{itemize}
\item Area of a triangle with vertices $(x_1,y_1)$, $(x_2,y_2)$ and $(x_3,y_3)$ is
$\Delta=\frac{1}{2}\begin{vmatrix}
x_1&y_1&1\\
x_2&y_2&1\\
x_3&y_3&1
\end{vmatrix}$.
\item Minor of $a_{ij}$ is obtained by deleting the $i^{th}$ row and $j^{th}$ column and is denoted by $M_{ij}$.
\item Cofactor of $a_{ij}$ is $A_{ij}=(-1)^{i+j}M_{ij}$.
\item Determinant value equals the sum of products of elements of a row/column with corresponding cofactors, e.g.
$|A|=a_{11}A_{11}+a_{12}A_{12}+a_{13}A_{13}$.
\item Sum of products of elements of one row with cofactors of another row is zero.
\item For
$A=\begin{pmatrix}
a_{11}&a_{12}&a_{13}\\
a_{21}&a_{22}&a_{23}\\
a_{31}&a_{32}&a_{33}
\end{pmatrix}$,
the adjoint is
$adj\,A=\begin{pmatrix}
A_{11}&A_{21}&A_{31}\\
A_{12}&A_{22}&A_{32}\\
A_{13}&A_{23}&A_{33}
\end{pmatrix}$.
\item $A(adj\,A)=(adj\,A)A=|A|I$.
\item $A$ is singular if $|A|=0$ and non-singular if $|A|\neq0$.
\item If $AB=BA=I$, then $B$ is the inverse of $A$ and $A^{-1}=B$.
\item $A$ has an inverse iff it is non-singular.
\item $A^{-1}=\frac{1}{|A|}(adj\,A)$.
\item If
$a_1x+b_1y+c_1z=d_1$,
$a_2x+b_2y+c_2z=d_2$,
$a_3x+b_3y+c_3z=d_3$,
then the equations can be written as $AX=B$.
\end{itemize}
\end{tcolorbox}
\newpage
\begin{tcolorbox}[
colback=cyan!20,
colframe=cyan!60!black,
sharp corners,
boxrule=0.8pt,
left=18pt,right=18pt,top=12pt,bottom=12pt
]
\small
\begin{itemize}
\item Unique solution of $AX=B$ is $X=A^{-1}B$, where $|A|\neq0$.
\item A system is consistent or inconsistent according as its solution exists or not.
\item For a square matrix $A$ in $AX=B$:
\begin{itemize}[label=--]
\item If $|A|\neq0$, a unique solution exists.
\item If $|A|=0$ and $(adj\,A)B\neq0$, no solution exists.
\item If $|A|=0$ and $(adj\,A)B=0$, the system may or may not be consistent.
\end{itemize}
\end{itemize}
\end{tcolorbox}


\end{document}
