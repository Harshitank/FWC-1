\documentclass[12pt]{article}

\usepackage{amsmath,amssymb}
\usepackage{graphicx}
\usepackage[none]{hyphenat}

\setlength{\parindent}{0pt}

\begin{document}

\begin{center}
\includegraphics[width=4.5cm]{iiitb_logo.png}\\
{\Large $Harshita\ N\ Kumar$}\\
$ID:\ COMETFWC052$\\
{\large $CBSE\ Class\ XII$}\\
$Task\ 2$
\end{center}

\begin{center}
{\Large $43^{rd}\ International\ Mathematical\ Olympiad$}\\
$IMO\ 2002$
\end{center}

\vspace{0.3cm}

\begin{enumerate}

\item $S$ is the set of all $(h,k)$ with $h,k$ non-negative integers such that $h+k<n$.
Each element of $S$ is colored red or blue, so that if $(h,k)$ is red then $(h',k')$ is also red for all $h' \le h,\ k' \le k$.

A type 1 subset of $S$ has $n$ blue elements with different first member and a type 2 subset of $S$ has $n$ blue elements with different second member.
Show that there are the same number of type 1 and type 2 subsets.

\item $BC$ is a diameter of a circle with center $O$.
$A$ is any point on the circle with $\angle AOC > 60^\circ$.
$EF$ is the chord which is the perpendicular bisector of $AO$.
$D$ is the midpoint of the minor arc $AB$.
The line through $O$ parallel to $AD$ meets $AC$ at $J$.
Show that $J$ is the incenter of triangle $CEF$.

\item Find all pairs of integers $m>2,\ n>2$ such that there are infinitely many positive integers $k$ for which
\[
k^m + k^2 - 1 \mid k^n + k - 1.
\]

\item The positive divisors of the integer $n>1$ are
\[
d_1 < d_2 < \cdots < d_k,
\]
so that $d_1=1,\ d_k=n$.
Let
\[
d = d_1d_2 + d_2d_3 + \cdots + d_{k-1}d_k.
\]
Show that $d<n^2$ and find all $n$ for which $d$ divides $n^2$.

\item Find all real-valued functions on the reals such that
\[
f(x+y)\big(f(x)+f(y)\big)=f(xy)+f(x)+f(y)
\]
for all real numbers $x,y$.

\item $n>2$ circles of radius $1$ are drawn in the plane so that no line meets more than two of the circles.
Their centers are $O_1,O_2,\dots,O_n$.
Show that
\[
\sum_{1\le i<j\le n} \frac{1}{O_iO_j} \le \frac{(n-1)\pi}{4}.
\]

\end{enumerate}

\end{document}
