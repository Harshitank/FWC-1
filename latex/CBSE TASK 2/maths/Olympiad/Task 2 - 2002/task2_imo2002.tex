\documentclass[12pt,fleqn]{article}

%================ PACKAGES =================%
\usepackage[a4paper,margin=1in]{geometry}
\usepackage{amsmath,amssymb}
\usepackage{graphicx}
\usepackage{enumitem}

\setlength{\parindent}{0pt}
\setlength{\mathindent}{0pt}

\begin{document}

%================ HEADING =================%
\begin{center}
\includegraphics[width=5cm]{iiitb_logo.png}\\[0.4cm]

{\Large \textbf{Harshita N Kumar}}\\
{\normalsize \textbf{ID: COMETFWC052}}\\[0.3cm]

{\large \textbf{CBSE Class XII}}\\
{\large \textbf{Task–2 : IMO 2002 Problems}}
\end{center}

\vspace{0.4cm}
\hrule
\vspace{0.6cm}

%================ QUESTIONS =================%
\begin{enumerate}[leftmargin=0pt, label=\arabic*., itemsep=12pt]

\item Let $S$ be the set of all $(h,k)$ with $h,k$ non-negative integers such that
\[
h + k < n .
\]
Each element of $S$ is colored red or blue, so that if $(h,k)$ is red and
\[
h' \le h,\; k' \le k,
\]
then $(h',k')$ is also red.

A type~1 subset of $S$ has $n$ blue elements with different first members, and
a type~2 subset of $S$ has $n$ blue elements with different second members.

Show that there are the same number of type~1 and type~2 subsets.

\item $BC$ is a diameter of a circle with center $O$.
$A$ is any point on the circle with
\[
\angle AOC > 60^\circ .
\]

$EF$ is the chord which is the perpendicular bisector of $AO$.
$D$ is the midpoint of the minor arc $AB$.
The line through $O$ parallel to $AD$ meets $AC$ at $J$.

Show that $J$ is the incenter of triangle $CEF$.

\item Find all pairs of integers $m > 2,\; n > 2$ such that there are infinitely many
positive integers $k$ for which
\[
k^n + k^2 - 1 \mid k^m + k - 1 .
\]

\item The positive divisors of the integer $n > 1$ are
\[
d_1 < d_2 < \cdots < d_k ,
\]
so that $d_1 = 1$ and $d_k = n$.

Let
\[
d = d_1 d_2 + d_2 d_3 + \cdots + d_{k-1} d_k .
\]

Show that
\[
d < n^2
\]
and find all $n$ for which $d$ divides $n^2$.

\item Find all real-valued functions on the reals such that
\[
f(x) + f(y)\bigl(f(f(x)) + f(y)\bigr)
= f(x - y) + f(xy + y)
\]
for all real $x,y$.

\item $n > 2$ circles of radius $1$ are drawn in the plane so that no line meets more than
two of the circles.

Their centers are $O_1, O_2, \dots, O_n$.
Show that
\[
\sum_{i<j} \frac{1}{O_i O_j}
\le \frac{(n-1)\pi}{4}.
\]

\end{enumerate}

\end{document}
