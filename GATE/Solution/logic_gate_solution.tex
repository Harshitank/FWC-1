\documentclass[12pt]{article}

\usepackage{graphicx}
\usepackage{xcolor}
\usepackage{geometry}
\usepackage{amsmath}
\usepackage{enumitem}
\geometry{a4paper, margin=1in}

\begin{document}

\begin{center}
\includegraphics[width=0.3\textwidth]{iiitb_logo.png.jpg}
\end{center}

\vspace{0.5cm}

\begin{center}
\textbf{\Large Harshita N Kumar} \\[0.2cm]
\textbf{ID: COMETFWC052}
\end{center}

\vspace{1cm}

{\color{blue} \section*{QUESTION}}

The logic gate circuit shown in the adjoining figure realizes which of the following function?

\begin{enumerate}
\item XOR
\item XNOR
\item Half Adder
\item Full Adder
\end{enumerate}

\vspace{0.5cm}

\begin{center}
\includegraphics[width=0.8\textwidth]{IMG_20260216_144417.jpg}
\end{center}

\vspace{1cm}

{\color{blue} \section*{SOLUTION}}

\textbf{Step 1: Identification of Gates}

From the circuit diagram, all the gates used are NAND gates.  
A NAND gate performs the Boolean operation:

$A \text{ NAND } B = \overline{A \cdot B}$

Since NAND is a universal gate, any Boolean function can be implemented using only NAND gates.

\bigskip

\textbf{Step 2: First Stage Operation}

If both inputs of a NAND gate are same, it acts as an inverter:

$X \text{ NAND } X = \overline{X}$

$Y \text{ NAND } Y = \overline{Y}$

Thus, we obtain complemented inputs:

$\overline{X}$ and $\overline{Y}$

\bigskip

\textbf{Step 3: Intermediate Stage}

The cross-connected NAND gates generate:

$P = \overline{X \cdot \overline{Y}}$

$Q = \overline{\overline{X} \cdot Y}$

\bigskip

\textbf{Step 4: Final Stage}

The final NAND gate gives:

$Z = \overline{P \cdot Q}$

Substituting values of $P$ and $Q$:

$Z = \overline{ \left( \overline{X \cdot \overline{Y}} \right) \cdot \left( \overline{\overline{X} \cdot Y} \right) }$

Using De Morgan’s Theorem and Boolean simplification:

$Z = X\overline{Y} + \overline{X}Y$

\bigskip

\textbf{Step 5: Final Result}

The expression $Z = X\overline{Y} + \overline{X}Y$ is the standard form of the Exclusive-OR function.

Therefore,

$Z = X \oplus Y$

\bigskip

\textbf{Conclusion:}

The given circuit realizes the XOR function.

\bigskip

\textbf{Correct Answer: (a) XOR}

\end{document}
